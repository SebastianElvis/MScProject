\documentclass[11pt]{article}
\usepackage{fullpage}
%\usepackage{doublespace}
\begin{document}
\title{Weekly Status Report}
\author{Project Name \\
Name(s) of Participants \\
\date{Week Ending 8/29/03}
}
\maketitle
%
% This section is used to list the key action items from the
% previous meeting. This information will help provide 
% continuity of information and decisions made in the
% previous meeting. 
% Use the \item construct to list each item.  Try to keep the
% descriptions for each down to one or two sentences
%
\section{Summary of Action Items from Previous Meeting}
\begin{enumerate}
\item Action item description and status (completed, in progress, terminated) 
\item Action item
\end{enumerate}
%
% This section is used to list the accomplishments of the week.
% Use the \item construct to list each item.  Try to keep the
% descriptions for each down to one or two sentences
%
\section{Planned Accomplishments}
\begin{enumerate}
\item New item: give internal deadline for deliverable.
\item Person and Scheduled Task name:
   \begin{itemize}
      \item hours spent by person
      \item description of what was done (task doesn't need to be complete)
   \end{itemize}
\item Scheduled Task name:
   \begin{itemize}
      \item hours spent by person
      \item description of what was done (task doesn't need to be complete)
   \end{itemize}
\end{enumerate}

%
% This section is used to list the unscheduled accomplishments of the week.
% Use the \item construct to list each item.  Try to keep the
% Descriptions for each down to one or two sentences
%
\section{Other Accomplishments}
\begin{itemize}
\item
\item
\end{itemize}

%
% This section is used to list the following week's plan
% Use the \item construct to list each item.  Try to keep the
% Descriptions for each down to one or two sentences
%
\section{Next Week's Plan}
\begin{itemize}
\item
\item
\end{itemize}

%
% This section is used to list any issues that were raised during
% the week that are of special interest.  Also use this to voice
% issues that the TA or Instructor must resolve.
% Use the \item construct to list each item.  Try to keep the
% Descriptions for each down to one or two sentences
%
\section{Issues}
\begin{itemize}
\item
\item
\end{itemize}

\end{document}
